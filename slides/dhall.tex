\documentclass[unicode,12pt]{beamer}
\usepackage{luatexja}
\usepackage[hiragino-pro]{luatexja-preset}
\usepackage{luatexja-fontspec}
\renewcommand{\kanjifamilydefault}{\gtdefault}

\setmonofont{x14y24pxHeadUpDaisy}
\newfontfamily{\magnolia}{Magnolia Script}
\newfontfamily{\brush}{Brush Script MT}
\newfontfamily{\doulos}{Doulos SIL}

\usepackage{luatexja-ruby}
\usepackage{url}
\usepackage{listings}

\lstdefinelanguage{dhall}{
  keywords={let, in},
  keywordstyle=\color{main}\bfseries,
  keywords=[2]{Bool, Integer, Double, Text, List, Optional},
  keywordstyle=[2]\color{accent}\bfseries,
  identifierstyle=\color{text},
  sensitive=false,
  comment=[l]{--},
  %morecomment=[s]{{-}{-}},
  commentstyle=\color{subtext}\ttfamily,
  stringstyle=\color{main}\ttfamily,
  morestring=[b]',
  morestring=[b]"
}

\lstdefinelanguage{yaml}{
  keywords={true,false,yes,no},
  keywordstyle=\color{main}\bfseries,
  %basicstyle=\ttfamily,
  identifierstyle=\color{text},
  sensitive=false,
  comment=[l]{\#},
  %morecomment=[s]{{-}{-}},
  commentstyle=\color{main}\ttfamily,
  stringstyle=\color{main}\ttfamily,
  moredelim=**[il][\color{accent}:\color{subtext}]{:},
  morestring=[b]',
  morestring=[b]"
}

\lstset{
   language=dhall,
   extendedchars=true,
   basicstyle=\footnotesize\ttfamily,
   showstringspaces=false,
   showspaces=false,
   tabsize=2,
   breaklines=true,
   showtabs=false
}

\definecolor{main}{RGB}{49,118,137}
\definecolor{accent}{RGB}{220,94,74}
\definecolor{text}{RGB}{50,50,50}
\definecolor{subtext}{RGB}{80,80,80}

\usetheme{Copenhagen}
\usecolortheme{beaver}
\setbeamertemplate{footline}[page number]

\setbeamercolor{alerted text}{fg=accent}
\setbeamercolor{background canvas}{bg=white}
\setbeamercolor{block body alerted}{bg=normal text.bg!90!black}
\setbeamercolor{block body}{bg=normal text.bg!90!black}
\setbeamercolor{block body example}{bg=normal text.bg!90!black}
\setbeamercolor{block title alerted}{use={normal text,alerted text},fg=alerted text.fg!75!normal text.fg,bg=normal text.bg!75!black}
\setbeamercolor{block title}{bg=main}
\setbeamercolor{block title example}{use={normal text,example text},fg=example text.fg!75!normal text.fg,bg=normal text.bg!75!black}
\setbeamercolor{fine separation line}{}
\setbeamercolor{frametitle}{fg=main}
\setbeamercolor{item projected}{fg=subtext}
\setbeamercolor{normal text}{fg=subtext}
\setbeamercolor{palette sidebar primary}{use=normal text,fg=normal text.fg}
\setbeamercolor{palette sidebar quaternary}{use=structure,fg=structure.fg}
\setbeamercolor{palette sidebar secondary}{use=structure,fg=structure.fg}
\setbeamercolor{palette sidebar tertiary}{use=normal text,fg=normal text.fg}
\setbeamercolor{section in sidebar}{fg=brown}
\setbeamercolor{section in sidebar shaded}{fg=grey}
\setbeamercolor{separation line}{}
\setbeamercolor{sidebar}{bg=red}
\setbeamercolor{sidebar}{parent=palette primary}
\setbeamercolor{structure}{bg=white, fg=main}
\setbeamercolor{subsection in sidebar}{fg=brown}
\setbeamercolor{subsection in sidebar shaded}{fg=grey}
\setbeamercolor{title}{fg=main}
\setbeamercolor{titlelike}{fg=main}

\title{Dhall: Haskellの新たなキラーアプリ}
\author[@syocy]{
    \frame{\includegraphics[width=1cm]{icon.jpg}}\\%
    \vspace{0.5em}%
    @syocy%
}

\date{2018-11-10}

\begin{document}

\begin{frame}[plain]\frametitle{}
  \titlepage
\end{frame}

\section{Dhallとは}

\begin{frame}{設定ファイルとは}
  \begin{itemize}
  \item コンパイルなどの前処理をせずにプログラムのパラメータを変えたい
  \item → パラメータをソースコードの外に切り離したい
  \item → \alert{設定ファイル}
  \end{itemize}
\end{frame}

\begin{frame}{既存の設定ファイル言語への不満}
  \begin{itemize}
  \item 現在主流の設定ファイル言語は機能が少ない
    \begin{itemize}
    \item 同じ値を繰り返し書きたくない(Don't Repeat Yourself)
    \item 入ってはいけない変な値が入っていたら教えてほしい(静的検査, 型)
    \item 大きい設定ファイルを分割したい(インポート)
    \end{itemize}
  \item ただし、設定ファイルとしての領分は守ってほしい
    \begin{itemize}
    \item 副作用を持っていて動作が不定だったり、
    \item 無限ループを起こしてハングしたりはしないでほしい
    \end{itemize}
  \end{itemize}

  なにかいいものはないのか
\end{frame}

\begin{frame}[plain]\frametitle{}
  \centering
  \vspace{2em}
  \Huge\fontsize{60}{70}\brush\textcolor{text}{Dhall}
\end{frame}

\begin{frame}{Dhallとは}
  \begin{itemize}
    \item 今まで挙げた特徴を備えた設定ファイル言語
      \begin{itemize}
      \item \alert{型・関数・インポート}の機能がある
      \item 無限ループは起こらない(チューリング完全ではない)
      \item 副作用は(標準では)ない
      \end{itemize}
    \item 読みは {\doulos dɔl} (US) もしくは {\doulos dɔːl} (UK)
      \begin{itemize}
      \item カタカナでは「ダール」もしくは「ドール」?
      \item このスライドでは「ダール」と発音する
      \end{itemize}
  \end{itemize}
\end{frame}

\begin{frame}{Dhallの開発体制}
  \begin{itemize}
    \item 言語仕様は独立して管理されている
      \begin{itemize}
      \item \url{https://github.com/dhall-lang/dhall-lang}
      \end{itemize}
    \item 主要ツール、および言語仕様の参照実装はHaskellで書かれている
      \begin{itemize}
      \item \url{https://github.com/dhall-lang/dhall-haskell}
      \end{itemize}
  \end{itemize}
\end{frame}

\begin{frame}{Dhallの型: プリミティブな型}
  \centering
  \begin{tabular}{lll} \hline
    型 & 意味 & リテラル(例)\\ \hline
    \texttt{Bool} & 真偽値 & \texttt{True}, \texttt{False}\\
    \texttt{Natural} & 0以上の整数 & \texttt{0}, \texttt{1}\\
    \texttt{Integer} & 符号付き整数 & \texttt{-1}, \texttt{+1}\\
    \texttt{Double} & 小数点付き数 & \texttt{0.1}, \texttt{-2e10}\\
    \texttt{Text} & 文字列 & \texttt{""}, \texttt{"abc"}\\ \hline
  \end{tabular}

  \begin{itemize}
  \item プリミティブ型には一般的なものが揃っている
  \item 数値の型はできるだけ \texttt{Natural} を使うのがおすすめ
    \begin{itemize}
    \item 使える標準関数が多いため
    \end{itemize}
  \end{itemize}
\end{frame}

\begin{frame}{Dhallの型: 複合型}
  \centering
  \begin{tabular}{lll} \hline
    型 & 意味 & 値の例\\ \hline
    \texttt{List} & 0個以上の値の集まり & \lstinline|[1,2,3]| \\
    \texttt{Optional} & 0〜1個の値の集まり & \lstinline|Some 1| \\
    Record & キーと値のペアの集まり & \lstinline|{a=1, b=2}| \\
    Union & 「どれかひとつ」を表す & {\footnotesize\texttt{<A=\{=\}|B:\{\}>}} \\ \hline
  \end{tabular}

  \begin{itemize}
  \item RecordはJSONのオブジェクトに相当
  \item Unionの書き方が特徴的
    \begin{itemize}
    \item 上記の例では「AとBのラベルがあるUnion型でAを選んだ値」となる
    \item 記述をサポートする標準関数がある
    \end{itemize}
  \end{itemize}
\end{frame}

\begin{frame}{インポート}
  ローカルパスとURLからのインポートができる
  \begin{itemize}
  \item ローカルパスからのインポート
    \begin{itemize}
    \item \texttt{let config = ./my/local/config.dhall}
    \end{itemize}
  \item URLからのインポート
    \begin{itemize}
    \item \texttt{let config = https://example.com/config.dhall}
    \item インポートにあたってハッシュ値のチェックをすることもできる
    \end{itemize}
  \item Dhallファイルではないraw textのインポートも可能
    \begin{itemize}
    \item 長い自然言語文章などに
    \end{itemize}
  \end{itemize}
\end{frame}

\begin{frame}{Dhallの導入は難しい?}
  \begin{itemize}
  \item なるほどDhallは便利かも
  \item だけど自分の使っている言語にバインディングがない\footnote{今のところ最新仕様に準拠したバインディングはHaskellのみ}
  \item もうYAMLやJSONで設定作ってしまっているし……
  \end{itemize}

  本当に導入は難しいのか
\end{frame}

\begin{frame}{Dhallの導入は簡単}
  \begin{itemize}
  \item \alert{dhall-to-yaml/dhall-to-json}
    \footnote{\url{https://github.com/dhall-lang/dhall-lang/wiki/Getting-started\%3A-Generate-JSON-or-YAML}}
    がある
    \begin{itemize}
      \item DhallファイルをYAMLやJSONに変換するコマンドラインツール
      \item プログラムでDhallを読み込むことを考えなくて良い
      \item バインディングがない言語でもDhallを使える
      \item 導入はMac, Linuxはコマンド一発、WindowsはThe Haskell Tool Stackを入れる必要がある
    \end{itemize}
  \end{itemize}
\end{frame}

\section{例題: KubernetesのYAMLを書いてみる}

\begin{frame}{例題: KubernetesのYAMLを書いてみる}
  \begin{itemize}
  \item 最近話題のOSS, KubernetesはYAMLを大量に用いることで有名
    \begin{itemize}
    \item Wall of YAMLなどとも呼ばれる
    \item 実際さまざまなYAML管理のソリューションが存在する
    \end{itemize}
  \item 今回は{\ttfamily dhall-to-yaml}を用いて安全・便利にKubernetesのYAMLを生成してみる
    \begin{itemize}
    \item 実はすでにdhall-kubernetes\footnote{\url{https://github.com/dhall-lang/dhall-kubernetes}}というのがあり、本当にKubernetes YAML
      を作るならそれを使う方がよい。今回はあくまで例題としてDIYする。
    \end{itemize}
  \end{itemize}
\end{frame}

\begin{frame}{目的とするYAML}{{\ttfamily service.yaml}\footnote{\url{https://kubernetes.io/docs/concepts/services-networking/service/}}}
  \lstinputlisting[breaklines=false,numbers=left,language=yaml]{../dhall/service.yaml}
\end{frame}

\begin{frame}{愚直なDhall}{\ttfamily k\_notype.dhall}
  まず型とかは考えずに愚直に書いてみる
  \lstinputlisting[breaklines=false,numbers=left,language=dhall]{../dhall/k_notype.dhall}
\end{frame}

\begin{frame}{愚直なDhallをYAMLに変換}
  目的とするYAMLができた!
  \lstinputlisting[breaklines=false,numbers=left,language=yaml]{../dhall/k_notype.dhall.yaml}
\end{frame}

\begin{frame}{よりDhallらしい方法}
  \begin{itemize}
  \item 型などを意識しなくても目的とするYAMLはできた
    \begin{itemize}
    \item ユースケースによってはこれくらいでもまあ便利
    \end{itemize}
  \item さらにDhallの能力を引き出すなら、型やデフォルト値などを用意することができる
    \begin{itemize}
    \item 基本的なアイデアとしては、 
      \begin{itemize}
      \item Union型を用いて記述できる値を制限する
      \item Record型でデフォルト値を用意する
      \end{itemize}
    \end{itemize}
  \end{itemize}
\end{frame}

\begin{frame}{Kubernetes YAMLの型を定義(1/2)}{\ttfamily k\_types.dhall}
  Kubernetes YAMLの型を定義するファイルを作る
  \lstinputlisting[breaklines=false,numbers=left,language=dhall,firstline=1,lastline=13]{../dhall/k_types.dhall}
\end{frame}

\begin{frame}{Kubernetes YAMLの型を定義(2/2)}{\ttfamily k\_types.dhall}
  Unionで表現した{\ttfamily ApiVersion}などをYAMLのStringに戻す処理も必要(実装は省略)
  \lstinputlisting[breaklines=false,numbers=left,language=dhall,firstline=18,lastline=29]{../dhall/k_types.dhall}
\end{frame}

\begin{frame}{定義した型を利用して書き直す}{\ttfamily k\_service.dhall}
  \lstinputlisting[breaklines=false,numbers=left,language=dhall]{../dhall/k_service.dhall}
\end{frame}

\begin{frame}{型付き版をYAMLに変換}
  より安全にKubernetes YAMLが表現できた
  \lstinputlisting[breaklines=false,numbers=left,language=yaml]{../dhall/k_service.dhall.yaml}
\end{frame}

\begin{frame}{デフォルト値(1/2)}{\ttfamily k\_service\_default.dhall}
  Kubernetes YAMLのデフォルト値を作ってみる
  \lstinputlisting[breaklines=true,numbers=left,language=dhall,firstline=1,lastline=6]{../dhall/k_service_default.dhall}
  デフォルト値を用意すると、同じような値をたくさん作るときに間違いをしにくくなる
  (今回例示するのは1つ)
\end{frame}

\begin{frame}{デフォルト値(2/2)}{\ttfamily k\_service\_default.dhall}
  Recordは演算子 {\ttfamily ∧}で組み合わせることができる
  \lstinputlisting[breaklines=false,numbers=left,language=dhall,firstline=8,lastline=19]{../dhall/k_service_default.dhall}
\end{frame}

\begin{frame}{デフォルト値版をYAMLに変換}
  もちろんこれまでと同じくYAMLに変換できる
  \lstinputlisting[breaklines=false,numbers=left,language=yaml]{../dhall/k_service_default.dhall.yaml}
\end{frame}

\section{}

\begin{frame}[plain]{まとめ}
  \begin{itemize}
  \item Dhallは\alert{設定ファイルとしての限界}を突き詰めた言語
    \begin{itemize}
    \item インポート・型・関数などの機能を持ちながら、
    \item 副作用や無限ループの危険がない
    \end{itemize}
  \item \alert{YAML/JSONに変換できる}ため、既存資産に組み入れやすい
    \begin{itemize}
    \item Mac, Linuxにはバイナリ配布があり導入もしやすい
    \end{itemize}
  \item \alert{ユースケースに合わせたレベル}で利用できる
    \begin{itemize}
    \item インポートとデフォルト値があればいい
    \item ばりばり厳密に型を定義したい、など
    \end{itemize}
  \end{itemize}
\end{frame}

\section{補遺: より高度な使い方}

\begin{frame}{補遺: より高度な使い方}

\end{frame}

\end{document}
